\documentclass{article}
\usepackage[latin1]{inputenc}
\usepackage{enumerate}
\usepackage{hyperref}
\usepackage{graphics}
\usepackage{graphicx}
\usepackage{caption}
\usepackage{subcaption}
\usepackage{tabularx}
\usepackage{amsmath}
\usepackage{amssymb}
\newcommand{\ket}[1]{\ensuremath{\left|#1\right\rangle}}
\newcommand{\bra}[1]{\ensuremath{\left\langle#1\right|}}
\newcommand{\braket}[2]{\ensuremath{\left\langle #1 \middle| #2 \right\rangle}}
\newcommand{\obar}[1]{\ensuremath{\overline{ #1 }}}
\hypersetup{colorlinks=true, urlcolor=blue, linkcolor=blue, citecolor=red}
\graphicspath{ {C:/Users/Evan/Desktop/} }
\title{Determination of Magnitude and Intrinsic Flux of HD 227858 and HD 338931 from Landholt Standard SA 113475}
\author{Evan Ott \\ UT EID: eao466}
\date{\today}
\setcounter{secnumdepth}{0}
%\usepackage[parfill]{parskip}
\begin{document}
\maketitle
\section{Introduction}
While the casual astrophotographer might only care about seeing as many stars as possible in an image, in academic astronomy,
we often want to have quantifiable data about stars, for example the magnitude (measure of brightness) of the star to compare it
to others, either in terms of total light or broken into sections of the electromagnetic spectrum by filters. Or, we might want to
have the flux (photon energy or number per area each second). However, besides having proper equipment, much calibration
must take place to account for all present factors - atmospheric extinction of photons travelling at different angles
through Earth's atmosphere, ever-present and never-present pixels in our sensor, absorption in mirrors and lenses, etc. Here, I work through calculating the magnitude and intrinsic flux (flux just outside the atmosphere) of stars HD 227858
and HD 338931 based on the Landhold Standard star SA 113475{\huge\cite{lanholt}}. 

\begin{table}
\begin{center}
\begin{tabular}{c | c | c | c}
Filter & $\lambda_0$ & $\Delta\nu$ & Absolute Spectral Radiance  \\
 & [$\mu$m]& [Hz]& [erg cm$^{-2}$s$^{-1}$\AA$^{-1}$] \\
\hline
B & 0.44 & 3.06E15 & 6.60E-9 \\
V & 0.55& 3.37E15 &3.64E-9 \\
R & 0.70& 1.36E15 &1.36E-9 \\
I & 0.90& 6.81E14&8.30E-10
\end{tabular}
\end{center}
\caption{Johnson Filter Band parameters, with $\lambda_0$ as the central wavelenth,
$\Delta\nu$ as the FWHM of frequency of the filter and absolute spectral radiance as the
flux per wavelength to be observed through the filter for a 0-magnitude star in the band.}
\label{table:filters}
\end{table}

\section{Methods}
Using the 16" telescope in Robert Lee Moore Hall at the University of Texas at Austin, I and my classmates used a 1024x1024 CCD imaging
system, coupled with a selection of Johnson-type filters for the B, V, R, and I parts of the spectrum. Parameters of the filters can be found in Table
\ref{table:filters}. After taking dark frame and flat frame images (to account for the CCD's inherent background response to temperature and
non-uniform sensitivity, respectively), we were able to use the IRAF suite of programs to isolate the count (in Arbitrary Data Units [ADU])
in the image due to photons from the stars of interest and the count associated with background light from other sources. These data
are found in Table \ref{table:Counts}.

\begin{table}
\begin{center}
\begin{tabular}{c | c | c | c | c | c }
Star & Filter & t & $X=\sec{z}$ & $S_{*,sky}$ & $S_*$\\
 &  & [s]&  & $\mathsf{SUM}$ & $\mathsf{FLUX=SUM-MSKY*AREA}$\\
\hline
HD 227858 & B & 30 & 1.294 & 117809 & 39381 \\
HD 227858 & V & 30 & 1.301 & 388133 & 131079  \\
HD 227858 & R & 30 & 1.307 & 567655 & 254168  \\
HD 227858 & I &  30 & 1.323 & 417430 & 260971   \\
HD 338931 & B & 30 & 1.394 & 255793 & 152570 \\
HD 338931 & V & 30 & 1.405 & 767017 & 494401  \\
HD 338931 & R & 30 & 1.413 & 1390626 &  977589 \\
HD 338931 & I & 30 &  1.424 & 1285304 &  1075136 \\
SA 113475 & B & 30 & 1.206  & 169993 & 38822  \\
SA 113475 & B & 30 & 1.482  & 179149 &  33471 \\
SA 113475 & V & 30 & 1.207  & 477466 & 154874 \\
SA 113475 & V & 30 & 1.498  & 665531 &  147737 \\
SA 113475 & R & 30 & 1.209 & 802783  & 344417  \\
SA 113475 & R & 30 & 1.498 & 1064716 & 332641 \\
SA 113475 & I & 30 & 1.213 & 477466  & 154874  \\
SA 113475 & I & 30 & 1.521& 665531 & 147737   \\
\end{tabular}
\caption{Signal for each target star as extracted from IRAF's ${\texttt{apphot phot}}$ function, presented with
signal including background signal ($S_{*,sky}$), and signal with sky removed ($S_*$). $t$ is the exposure time in seconds,
and $X$ is the airmass given as the secant of the zenith angle of the observation.}
\label{table:Counts} 
\end{center}
\end{table}

To determine the relative magnitudes of the target stars, it is sufficient to account for two factors: atmospheric extinction
and airmass. Airmass is simply a measure of how much of the atmosphere relative to the most direct path ($z=0$) the photon
had to travel and atmospheric extinction is a loss of photons due to absorption in the atmosphere before reaching the telescope.
For a single star $a$ measured once with airmass $X$, and exposure time $t$, giving rise to signal $S$, then measured again
with $X',~t'$ giving $S'$, we have the relation given in Equation \ref{eq:ext},

\begin{equation}
\label{eq:ext}
m_a-m_a'=k(X_a-X_a')=-2.5\log_{10}\frac{S_at_a'}{S_a't_a}
\end{equation}

where $m_a$ and $m_a'$ are the associated apparent magnitudes of the star in the different positions in the sky and $k$ is the
extinction coefficient (in the filter band used). From the $S-X$ data in Table \ref{table:Counts} for our standard star, we can
calculate the extinction coefficients for each filter band (see Table \ref{table:coeff}).

From Equation \ref{eq:ext} and reference data for SA 113475, we arrive at the extinction coefficients in Table \ref{table:coeff}.

\begin{table}
\begin{center}
\begin{tabular}{c | c}
Filter & k \\
\hline
B& 0.583\\
V&0.176\\
R& 0.131\\
I&0.166
\end{tabular}
\end{center}
\caption{Extinction coefficients determined from the two observations of SA 113475 as seen in Table \ref{table:Counts} using Equation \ref{eq:ext}.}
\label{table:coeff}
\end{table}

If we generalize Equation \ref{eq:ext} to account for comparing two different stars (which would not in general have the same absolute brightness),
we arrive at Equation \ref{eq:targetstar}. Equation \ref{eq:targetstar} allows for the simple calculation of $m_*$, the magnitude
of the target star, given information already known with the sole addition of the magnitude of the reference star, which is found in Table
\ref{table:SAmag}.

\begin{equation}
\label{eq:targetstar}
m_*-m_r=k(X_r-X_*)-2.5\log_{10}\frac{S_*t_r}{S_rt_*}
\end{equation}

\begin{table}
\begin{center}
\begin{tabular}{c | c}
Filter & Intrinsic Magnitude \\
\hline
B & 11.362 \\
V &  10.304\\
R & 9.736\\
I & 9.208
\end{tabular}
\end{center}
\caption{Intrinsic magnitude of SA 113475 from \huge FIXME FIXME FIXME}
\label{table:SAmag}
\end{table}

Applying Equation \ref{eq:targetstar} gives us the intrinsic magnitude of each of our target stars as seen in Table \ref{table:relmag}.

\begin{table}
\begin{center}
\begin{tabular}{c | c | c }
Star & Filter & $m_*$\\
\hline
HD 227858 & B & +11.295\\
HD 227858 & V & +10.469 \\
HD 227858 & R & +10.053 \\
HD 227858 & I &   +8.623 \\
HD 338931 & B & +9.766 \\
HD 338931 & V & +9.009\\
HD 338931 & R &  +8.576\\
HD 338931 & I & +7.018
\end{tabular}
\end{center}
\caption{Based on Table \ref{table:SAmag} and Equation \ref{eq:targetstar}, the intrinsic magnitudes of the target stars.}
\label{table:relmag}
\end{table}

To calculate flux of the target stars, we need the flux from a reference star. While our reference star was able to provide enough information to relate magnitudes,
these magnitudes, even coupled with signals are insufficient to provide more than a ratio. However, the ``absolute spectral irradiance''
aspect of the filter is defined to be the physical flux that would pass through the filter for a 0-magnitude source. Thus, rewriting Equation
\ref{eq:targetstar} to solve for flux, given $f_0$ for the filter, we can calculate the flux of our reference star with Equation \ref{eq:zeroflux}. 
The measured flux for SA 113475 is given in Table \ref{table:SAflux}.

\begin{equation}
\label{eq:zeroflux}
f_{ref}=f_010^{-(m_{ref}+k(X_{ref}-1))/2.5}
\end{equation}

\begin{table}
\begin{center}
\begin{tabular}{c | c}
Filter & Measured Flux \\
& [erg cm$^{-2}$s$^{-1}$]\\
\hline
B & 7.44E-10\\
V & 1.46E-9\\
R & 1.18E-9\\
I & 1.5E-9
\end{tabular}
\caption{Flux for SA 113475 by filter for $X\approx1.21$, calculated with Equation \ref{eq:zeroflux} from Table \ref{table:filters}
and Table \ref{table:SAmag}, and extinction coefficients in Table \ref{table:coeff}.}
\label{table:SAflux}
\end{center}
\end{table}

We can now do some algebraic sleuthing to determine how our telescope reacts to light with each filter. For a given filter,
we have that $S_*=\frac{\Delta\nu{\obar{\Phi}}t{\obar{f_*}}A}{g}$ where $\Delta\nu$ is the FWHM of the filter in frequency,
$t$ is the exposure time, $A$ is the aperture of the telescope, $g$ is the system gain, $f_*$ is the measured flux
of the star, $S_*$ is the measured signal from the star and $\obar\Phi$ is the averaged impact of losses due to reflection, transmission, etc.
By converting the flux from an energy flux to photon flux then that photon flux to number of photons per area per time (see Equation
\ref{eq:signalflux}), we can calculate $R$, the system response, which is the average number of counts generated per photon in the filter band. 
With the signal data from Table \ref{table:Counts} for SA 113475, 
coupled with Equation \ref{eq:signalflux}, we arrive at the signal responses in \ref{table:response}.

\begin{equation}
\label{eq:signalflux}
S_*=\frac{\Delta\nu{\obar{\Phi}}\frac{hc}{\lambda_0}}{g}n_\gamma=R\cdot n_\gamma
\end{equation}

\begin{table}
\begin{center}
\begin{tabular}{c | c}
Filter & Signal Response \\
& [count photon$^{-1}$]\\
\hline
B & 0.0061\\
V & 0.0100\\
R & 0.0216\\
I & 0.0076
\end{tabular}
\caption{Signal response for each Johnson-type filter in ADU count per photon.}
\label{table:response}
\end{center}
\end{table}

A handy consequence of Equation \ref{eq:signalflux} is that using the same filter and telescope setup for two
stars at the same exposure time gives $\frac{S_1}{S_2}=\frac{f_1}{f_2}$. As such, we can easly calculate the measured flux
of each target star for each filter (see Table \ref{table:fluxes}).

% $A=1280cm^2$ for the telescope, $t=30s$ for our dataset, and 
\begin{table}
\begin{center}
\begin{tabular}{c | c | c}
Star& Filter & Measured Flux\\
& &[erg cm$^{-2}$s$^{-1}$]\\
\hline
HD 227858 & B & 7.55E-10\\
HD 227858 & V & 1.24E-9\\
HD 227858 & R & 8.71E-10\\
HD 227858 & I & 2.53E-9\\
HD 338931 &B & 2.92E-9\\
HD 338931 &V & 4.66E-9\\
HD 338931 &R & 3.35E-9\\
HD 338931 & I & 1.04E-8\\
\end{tabular}
\caption{FIXME}
\label{table:fluxes}
\end{center}
\end{table}


\section{Results}
Assuming $g=3.0e^-/ADU$, and that each incoming photon that is not scattered, absorbed, or reflected energizes one electron,
the flux at the top of the telescope should be 3.0 times that as measured. If we apply Equation \ref{eq:realflux}, which expresses the
non-atmospherically-extincted flux as a function of airmass and measured flux (in this case, just above the telescope), we arrive at the
intrinsic flux for each star as stated in Table \ref{table:realflux}.

\begin{equation}
\label{eq:realflux}
F=f10^{\frac{kX}{2.5}}
\end{equation}

\begin{table}
\begin{center}
\begin{tabular}{c | c | c | c}
Star & Filter & Intrinsic Magnitude & Intrinsic Flux \\
& & & [erg cm$^{-2}$s$^{-1}$] \\
\hline
HD 227858 & B & 11.295 & 1.29E-8\\
HD 227858 & V & 10.469  & 6.30E-9\\
HD 227858 & R & 10.053 & 3.88E-9\\
HD 227858 & I  & 8.623 & 1.26E-8\\
HD 338931 & B & 9.766 & 5.69E-8\\
HD 338931 & V & 9.009  & 2.47E-8\\
HD 338931 & R & 8.576 & 1.54E-8\\
HD 338931 & I & 7.018 & 5.38E-8\\
SA 113475 & B & 11.362 & 1.13E-8\\
SA 113475 & V & 10.304 & 7.14E-9\\
SA 113475 & R  & 9.736 & 5.10E-9\\
SA 113475 & I  & 9.208 & 7.15E-9\\
\end{tabular}
\caption{FIXME}
\label{table:realflux}
\end{center}
\end{table}


\section{Analysis}





\section{Conclusion}






\begin{thebibliography}{9}

\end{thebibliography}



\end{document}