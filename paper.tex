\documentclass{article}
\usepackage[latin1]{inputenc}
\usepackage{enumerate}
\usepackage{hyperref}
\usepackage{graphics}
\usepackage{graphicx}
\usepackage{caption}
\usepackage{subcaption}
\usepackage{tabularx}
\usepackage{amsmath}
\usepackage{amssymb}
\newcommand{\ket}[1]{\ensuremath{\left|#1\right\rangle}}
\newcommand{\bra}[1]{\ensuremath{\left\langle#1\right|}}
\newcommand{\braket}[2]{\ensuremath{\left\langle #1 \middle| #2 \right\rangle}}
\newcommand{\obar}[1]{\ensuremath{\overline{ #1 }}}
\hypersetup{colorlinks=true, urlcolor=blue, linkcolor=blue, citecolor=red}
\graphicspath{ {C:/Users/Evan/Desktop/} }
\title{Relative Photometry of HD 227858 and HD 338931 from Landholt Standard SA 113475}
\author{Evan Ott \\ UT EID: eao466}
\date{\today}
\setcounter{secnumdepth}{0}
%\usepackage[parfill]{parskip}
\begin{document}
\maketitle
\section{Introduction}





\section{Methods}








\section{Results}

\begin{center}
\begin{tabular}{c | c | c | c | c | c }
							%SUM              FLUX
Star & Filter & t(s) & $X=\sec{z}$ & $S_{*,sky}=\mathsf{SUM}$ & $S_*=\mathsf{FLUX=SUM-MSKY*AREA}$\\
\hline
HD 227858 & B & 30 & 1.294 & 117809 & 39381 \\
HD 227858 & V & 30 & 1.301 & 388133 & 131079  \\
HD 227858 & R & 30 & 1.307 & 567655 & 254168  \\
HD 227858 & I &  30 & 1.323 & 417430 & 260971   \\

HD 338931 & B & 30 & 1.394 & 255793 & 152570 \\
HD 338931 & V & 30 & 1.405 & 767017 & 494401  \\
HD 338931 & R & 30 & 1.413 & 1390626 &  977589 \\
HD 338931 & I & 30 &  1.424 & 1285304 &  1075136 \\

SA 113475 & B & 30 & 1.206  & 169993 & 38822  \\
SA 113475 & B & 30 & 1.482  & 179149 &  33471 \\

SA 113475 & V & 30 & 1.207  & 477466 & 154874 \\
SA 113475 & V & 30 & 1.498  & 665531 &  147737 \\

SA 113475 & R & 30 & 1.209 & 802783  & 344417  \\
SA 113475 & R & 30 & 1.498 & 1064716 & 332641 \\

SA 113475 & I & 30 & 1.213 & 477466  & 154874  \\
SA 113475 & I & 30 & 1.521& 665531 & 147737   \\


\end{tabular}
\end{center}

From Equation \ref{eq:ext} and reference data for SA 113475, we arrive at the extinction coefficients:
\begin{align*}
k_B&= 0.583\\
k_V&= 0.176\\
k_R&= 0.131\\
k_I&= 0.166
\end{align*}
\begin{equation}
\label{eq:ext}
m_a-m_a'=k(X_a-X_a')=-2.5\log_{10}\frac{S_at_a'}{S_a't_a}
\end{equation}



With known magnitudes of a calibration star, we can use Equation \ref{eq:targetstar} to determine the magnitude of
the target star, $m_*$.
\begin{equation}
\label{eq:targetstar}
m_{rel,*}=m_*-m_r=k(X_r-X_*)-2.5\log_{10}\frac{S_*t_r}{S_rt_*}
\end{equation}
So relative to SA 113475, the target stars have relative magnitudes: FIXME add in intrinsic - it's ok
\begin{center}
\begin{tabular}{c | c | c }
Star & Filter & $m_{rel,*}$\\
\hline
HD 227858 & B & -0.067 \\
HD 227858 & V & +0.165 \\
HD 227858 & R & +0.317  \\
HD 227858 & I &   -0.585 \\

HD 338931 & B & -1.596 \\
HD 338931 & V & -1.295 \\
HD 338931 & R &  -1.160\\
HD 338931 & I & -2.190 \\
\end{tabular}
\end{center}

k tells us what the observed magnitude (m_obs = kX + m_int)

\section{Analysis}
$S_{*,sky}=\Delta\nu{\obar{\Phi}}{\obar{f}}tA/g$



FIXME
We can determine the precise relationship between flux and signal for each filter as well, using Equation \ref{eq:convert}
\begin{equation}
S_{*,sky}=\frac{\Delta\nu \obar{\Phi}t\obar{f}A}{g}
\label{eq:convert}
\end{equation}
where $\obar{f}$ is the averaged flux in the filter band, $t$ is the exposure time, $A$ is the aperture of the telescope (area at entry),
$g$ is the gain in electrons per count, $\obar{\Phi}=\obar{ql\phi}$ is the averaged response of the system over the filter, and $\Delta\nu$ is the
bandwidth of the filter.

\section{Conclusion}






\begin{thebibliography}{9}

\end{thebibliography}



\end{document}